\section{Prototype Implementation}
\label{sec:implementation}
\NewDocumentCommand{\codeword}{v}{%
\texttt{\textcolor{blue}{#1}}%
}

The prototype is implemented as a monorepo project. Meaning that both the
frontend and backend application are in the same parent folder/github
repository. Both the applications can be build and executed using gradle.

\subsection{Building and running the project}

There exist gradle tasks for building and starting the application. The steps
needed for this are:
\begin{enumerate}
	\item Build the frontend application and copy it to the correct location in the
	backend folder, so that the frontend will be hosted by the backend.
	Command: \codeword{gradle copyWebApp}
	\item Start the backend application. Command: \codeword{gradle bootRun}
\end{enumerate}

The backend depends on a RabbitMQ and Redis instance to be running. The ports
and URLs for this can be configured via environment variables
(\codeword{SPRING_REDIS_HOST}, \codeword{SPRING_REDIS_PORT},
\codeword{SPRING_RABBITMQ_HOST}, \codeword{SPRING_RABBITMQ_PORT}). To simplify
the development and deployment process there is a docker compose file in the
project directory which will automatically create and start Docker containers
for the backend application as well as containers for RabbitMQ and Redis.

For security reasons the project is configured in a way that the backend will
only accept reqeusts from the same origin, so starting the frontend application
separately via tools like \codeword{npm} and trying to connect to a different
instance of the backend application will not work.

\subsection{Authentication}

As the method of authentication and authorization GitHub OAuth2 is used. This
means a user will have to log in to their GitHub account in order to use the
FeedApp. Connecting a Spring application to an OAuth2 provider only requires
minimal configuration. All that is needed is adding the following dependency
and configuring some properties.\\
\codeword{implementation("org.springframework.boot:spring-boot-starter-oauth2-client")}
Since for this project GitHub was used as the provider the only properties to
configure were the client id, client secret and redirect URL configured in the
GitHub OAuth application.

\lstinputlisting{code/githubOauth.yml}

The following code snippet depicts the default behaviour of the FeedApp when the
user data is being retrieved. The FeedApp will try extracting the username from the
OAuth2 token that is used to authenticate the request. If no such username
exists in the token the controller will return an error (though this should never
happen since SpringSecurity will redirect the request to the GitHub
authentication page should no valid authentication exist). After extracting the
username the FeedApp will try extracting information about the user from the
database and create a new one if no user with the given username exists. Since the GitHub username 
is unique it is used as the Id for the User object. 

\lstinputlisting[language=java]{code/userController.java}
